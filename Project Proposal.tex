\documentclass[]{article}

%opening
\title{Project Proposal}
\author{Katie Winkle}

\begin{document}

\maketitle

\begin{abstract}

\end{abstract}

\tableofcontents

\section{Aims and Objectives}
\section{Motivation}
\section{Literature Review}
Lim et al. demonstrated a framework for dynamically mapping the emotion in a speech sample to robot gesturing \cite{lim2011converting}. Four parameters are identified that can be measured in the speech sample and applied to the robot's gesture; these are speed, intensity, regularity and extent (SIRE). For example, speed is measured by the speech rate of the voice sample and applied to the velocity of the gesture. The use of SIRE means that the emotional communication is pose-independent [contrasting with other research looking at specific gestures like arms up for surprise?]. Additionally, this is relatively simple(?) [compared to emotion generation models] because the robot requires no internal emotional state model. The framework was used to parameterise an example gesture on the NAO [more details on robot?] using actor speech samples and experiments were set up in order to test whether the resulting gesture successfully conveyed the emotional content of the original speech. The results suggest that changing the dynamics of a gesture, according to SIRE, can produce recognisable emotions at an inter-rater agreement of above 60\%. However, it was shown that in playing the original speech sample through the robot alongside the gesture had different impacts on different emotion, for example happiness was much easier to understand but anger was much harder. The authors hypothesised that this could be due to the neutral stance of the NAO, suggesting that whilst their method is designed to be independent of pose, the choice of gesture and pose is likely to be of importance in successfully conveying the desired emotion.


\section{Risk Register}
\section{Timeline}

\end{document}
